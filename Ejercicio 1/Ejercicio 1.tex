\documentclass[a4paper]{article}
\usepackage[utf8]{inputenc}
\usepackage[spanish, es-tabla]{babel}
\usepackage[table,xcdraw]{xcolor}
\usepackage{amsmath}
\usepackage{amsfonts}
\usepackage{amssymb}

\usepackage{float}
\usepackage{graphicx}

\usepackage{caption}
\usepackage{subcaption}
\captionsetup{compatibility=false}

\usepackage{multirow}
\setlength{\doublerulesep}{\arrayrulewidth}

\newcommand{\quotes}[1]{``#1''}
\newcommand\underrel[2]{\mathrel{\mathop{#2}\limits_{#1}}}

\usepackage{array}
\newcolumntype{C}[1]{>{\centering\let\newline\\\arraybackslash\hspace{0pt}}m{#1}}

\usepackage[american]{circuitikz}
\usepackage{xcolor}
\usepackage{fancyhdr}

\newlength{\stockheight}
\usepackage{hyperref}

\hypersetup{
    colorlinks=true,
    linkcolor=blue,
    filecolor=magenta,      
    urlcolor=blue,
    citecolor=blue,    
}

\urlstyle{same}

\usepackage{units} 
\pagestyle{fancy}
\fancyhf{}

\rfoot{Página \thepage}



\begin{document}
\subsection{Introducción}
En esta sección se implementó una compuerta \textbf{NOT} utilizando diversas tecnologías, siendo estas TTL (Transistor-Transistor-Logic), RTL (Resistor-Transistor-Logic) mediante transistores BJT (Bipolar Junction Transistor) y finalmente una variación de RTL utilizando un transistor MOSFET (Metal Oxide Semiconductor Field Efect Transistor).


\end{document}
