\documentclass[a4paper]{article}
\usepackage[utf8]{inputenc}
\usepackage[spanish, es-tabla]{babel}

\usepackage{amsmath}
\usepackage{amsfonts}
\usepackage{amssymb}

\usepackage{float}
\usepackage{graphicx}

\usepackage{caption}
\usepackage{subcaption}
\captionsetup{compatibility=false}

\usepackage{multirow}
\setlength{\doublerulesep}{\arrayrulewidth}

\newcommand{\quotes}[1]{``#1''}
\newcommand\underrel[2]{\mathrel{\mathop{#2}\limits_{#1}}}

\usepackage{array}
\newcolumntype{C}[1]{>{\centering\let\newline\\\arraybackslash\hspace{0pt}}m{#1}}

\usepackage[american]{circuitikz}
\usepackage{xcolor}
\usepackage{fancyhdr}

\newlength{\stockheight}
\usepackage{hyperref}

\hypersetup{
    colorlinks=true,
    linkcolor=blue,
    filecolor=magenta,      
    urlcolor=blue,
    citecolor=blue,    
}

\urlstyle{same}

\pagestyle{fancy}
\fancyhf{}
\lhead{22.13 Electrónica III}
\rhead{Mechoulam, Lambertucci, Martorell, Londero}
\rfoot{Página \thepage}
\input{../Informe/fillarea.tex}

\begin{document}
\subsection{Análisis de tecnologías}

En esta instancia del informe, se procede a comparar compuertas lógicas del tipo NOR de diversas tecnologías. Para ello se vale de las hojas de datos de las compuertas \href{http://www.ti.com/lit/ds/symlink/sn74hc02.pdf}{74HC02}, \href{http://www.ti.com/lit/ds/symlink/sn74hct02.pdf}{74HCT02} y \href{http://www.ti.com/lit/ds/symlink/sn74ls02.pdf}{74LS02}. Previo a dicho análisis, cabe detallar cada una de las tecnologías. Primero, se encuentra el 74HC02, siendo este, como su nombre lo indica, del tipo HC, cuyas siglas significan ``High-speed CMOS'', tecnología caracterizada por ser de baja potencia y alta velocidad. Luego se encuentra el 74HCT02, siendo HCT una variación de la tecnología HC. Esta denominación proviene de las mismas siglas que HC, solo que ademas posee lo que se conoce como ``logica transistor–transistor'' (TTL). En otras palabras, este tipo de compuertas puede operar bajo dicho estándar de tensiones, tanto de alimentación como de input.\footnote{``Logic family'', En.wikipedia.org, 2019. [Online]. Available: \url{https://en.wikipedia.org/wiki/Logic\_family\#HC\_logic}. [Accessed: 21- Sep- 2019].} Finalmente se encuentra el 74LS02, cuyas siglas provienen de ``Low-power Schottky''. Los integrados de esta familia se caracterizan por estar hechos con tecnología TTL.\footnote{``Serie 7400'', Es.wikipedia.org, 2019. [Online]. Available: \url{https://es.wikipedia.org/wiki/Serie\_7400}. [Accessed: 21- Sep- 2019].} Se destaca que este último, a diferencia de los dos primeros, se caracteriza por ser fabricado mediante el uso de tecnología BJT.

Analizando las respectivas hojas de dato, se recopila información sobre los  valores aceptables de señal, tanto de entrada como de salida. Es así que se realiza la siguiente tabla:
\begin{table}[H]
\centering
\begin{tabular}{c|c|c|c|c|c|c|c|}
\cline{2-8}
                               & $\mathbf{V_{CC}}$ \textbf{[V]} & \multicolumn{2}{c|}{\textbf{74HC02}}  & \multicolumn{2}{c|}{\textbf{74HCT02}} & \multicolumn{2}{c|}{\textbf{74LS02}} \\ \cline{3-8} 
                               &              & \textbf{Min. [V]} & \textbf{Max. [V]} & \textbf{Min. [V]} & \textbf{Max. [V]} & \textbf{Min. [V]} & \textbf{Max. [V]}	\\ \hline
\multicolumn{1}{|c|}{}         & 2            & 1.9           & -            & -             & -            & -          & -            \\  
\multicolumn{1}{|c|}{$\mathbf{V_{OH}}$} & 4.5          & 4.4           & -            & 3.84          & -            & 2.7            & -            \\  
\multicolumn{1}{|c|}{}         & 6            & 5.9           & -            & -             & -            & -            & -            \\ \hline
\multicolumn{1}{|c|}{}         & 2            & -             & 0.1          & -             & -            & -            & -          \\
\multicolumn{1}{|c|}{$\mathbf{V_{OL}}$} & 4.5          & -             & 0.1          & -             & 0.33         & -            & 0.5            \\
\multicolumn{1}{|c|}{}         & 6            & -             & 0.1          & -             & -            & -            & -            \\ \hline
\multicolumn{1}{|c|}{}         & 2            & 1.5           & -            & -             & -            & -            & -            \\ 
\multicolumn{1}{|c|}{$\mathbf{V_{IH}}$}  & 4.5          & 3.15          & -            & 2             & -            & 2            & -            \\ 
\multicolumn{1}{|c|}{}         & 6            & 4.2           & -            & -             & -            & -            & -            \\ \hline
\multicolumn{1}{|c|}{}         & 2            & -             & 0.5          & -             & -            & -            & -            \\ 
\multicolumn{1}{|c|}{$\mathbf{V_{IL}}$} & 4.5          & -             & 1.35         & -             & 0.8          & -            & 0.8          \\ 
\multicolumn{1}{|c|}{}         & 6            & -             & 1.8          & -             & -            & -            & -            \\ \hline
\end{tabular}
\centering
\caption{Tabla de valores de entrada y salida.}
\label{tabla:vinout}
\end{table}

Con la información que se ha detallado, se procede a analizar el margen de ruido, tanto para los niveles altos (high), como para los bajos (low), al combinar tecnologías HC y LS, siendo este calculado de la forma
\begin{equation*}
\left\{
\begin{aligned}
		& NM_{High}= V_{OH} - V_{IH} \\
		& NM_{Low}= V_{IL} - V_{OL} 
\end{aligned}
\right.
\end{equation*}

Nuevamente se decide plasmar los resultados en una tabla:
\begin{table}[H]
\centering
\begin{tabular}{|c|c|c|c|c|}
\hline
\textbf{In} & \textbf{Out} & $\mathbf{V_{CC}}$ \textbf{[V]} & $\mathbf{NM_{High}}$ \textbf{[V]} & $\mathbf{NM_{Low}} $\textbf{[V]} \\ \hline
74LS02      & 74HC02       & 4.5                            & 2.4                               & 0.7                              \\ 
74HC02      & 74LS02       & 4.5                              & -0.45                               & 0.85                                \\ \hline
\end{tabular}
\caption{Margen de ruido para combinaciones de tecnologías HC y LS.}
\label{tabla:nm}
\end{table}

Además, se busca plasmar de una forma más clara los datos obtenidos en la Tabla (\ref{tabla:vinout}). Esto se logra mediante los siguientes gráficos.  

\begin{figure}[H]
\begin{center}
\begin{tikzpicture}[yscale=1]
	\node [below] at (0,0) {Out: 74HC02};
	
	\draw[-][draw=gray, line width=2mm] (0,4.4) -- (0,4.5);
	\draw[-][draw=gray, line width=2mm] (0,0) -- (0,0.1);	
	\draw[-][draw=black, very thick] (-0.2,4.4) -- (0.2,4.4);
	\draw[-][draw=black, very thick] (-0.2,0.1) -- (0.2,0.1) node[label=right:0.1 V](){};

	\draw[-][draw=black, very thick] (0,0) -- (0,4.5);
	\draw[-][draw=black, very thick] (-0.1,2.25) -- (0.1,2.25) node[label=left:2.25 V](){};
	\draw[-][draw=black, very thick] (-0.2,0) -- (0.2,0);
	\draw[-][draw=black, very thick] (-0.2,4.5) -- (0.2,4.5);
	
	\draw[-][draw=black, very thick] (-0.4,4.5) -- (-0.6,4.5);
	\draw[-][draw=black, very thick] (-0.6,4.5) -- (-0.6,4.4);
	\draw[-][draw=black, very thick] (-0.4,4.4) -- (-0.6,4.4);
	\node [left] at (-0.6,4.45) {High};
	\node [above] at (0.2,4.5) {4.5 V};
	\node [right] at (0.2,4.4) {4.4 V};
	
	\draw[-][draw=black, very thick]	 (-0.4,0) -- (-0.6,0);
	\draw[-][draw=black, very thick]	 (-0.4,0.1) -- (-0.6,0.1);
	\draw[-][draw=black, very thick]	 (-0.6,0) -- (-0.6,0.1);
	\node [left] at (-0.6,0.05) {Low};
	
	
	\node [below] at (5,0) {In: 74LS02};
	\draw[-][draw=gray, line width=2mm] (5,0) -- (5,0.8);
	\draw[-][draw=gray, line width=2mm] (5,2) -- (5,4.5);
	\draw[-][draw=black, very thick] (5.2,0.8) -- (4.8,0.8) node[label=left:0.8 V](){};
	\node [above] at (5.2,4.5) {4.5 V};
	
	\draw[-][draw=black, very thick] (5,0) -- (5,4.5);
	\draw[-][draw=black, very thick] (4.9,2.25) -- (5.1,2.25);
	\draw[-][draw=black, very thick] (4.8,0) -- (5.2,0);
	\draw[-][draw=black, very thick] (4.8,4.5) -- (5.2,4.5);
	\draw[-][draw=black, very thick] (5.4,0) -- (5.6,0);
	\draw[-][draw=black, very thick] (4.8,2) -- (5.2,2);
	\draw[-][draw=black, very thick] (5.4,0.8) -- (5.6,0.8);
	\draw[-][draw=black, very thick] (5.6,0) -- (5.6,0.8);
	\node [right] at (5.6,0.4) {Low};

	\draw[-][draw=black, very thick] (5.4,2) -- (5.6,2);
	\node [left] at (4.8,2) {2 V};	
	\draw[-][draw=black, very thick] (5.4,4.5) -- (5.6,4.5);
	\draw[-][draw=black, very thick] (5.6,2) -- (5.6,4.5);
	\node [right] at (5.6,3.25) {High};	
\end{tikzpicture}
\end{center}
\caption{Comparación de tecnologías con HC a la salida y LS a la entrada.}
\label{fig:hcls}
\end{figure}

\begin{figure}[H]
\begin{center}
\begin{tikzpicture}[yscale=1]
	\node [below] at (0,0) {Out: 74LS02};
	
	\draw[-][draw=gray, line width=2mm] (0,2.7) -- (0,4.5);
	\draw[-][draw=gray, line width=2mm] (0,0) -- (0,0.5);	
	\draw[-][draw=black, very thick] (-0.2,2.7) -- (0.2,2.7);
	\draw[-][draw=black, very thick] (-0.2,0.5) -- (0.2,0.5) node[label=right:0.5 V](){};

	\draw[-][draw=black, very thick] (0,0) -- (0,4.5);
	\draw[-][draw=black, very thick] (-0.1,2.25) -- (0.1,2.25) node[label=left:2.25 V](){};
	\draw[-][draw=black, very thick] (-0.2,0) -- (0.2,0);
	\draw[-][draw=black, very thick] (-0.2,4.5) -- (0.2,4.5);
	
	\draw[-][draw=black, very thick] (-0.4,4.5) -- (-0.6,4.5);
	\draw[-][draw=black, very thick] (-0.6,4.5) -- (-0.6,2.7);
	\draw[-][draw=black, very thick] (-0.4,2.7) -- (-0.6,2.7);
	\node [left] at (-0.6,3.6) {High};
	\node [above] at (0.2,4.5) {4.5 V};
	
	\node [right] at (0.2,2.5) {2.7 V};
	
	\draw[-][draw=black, very thick]	 (-0.4,0) -- (-0.6,0);
	\draw[-][draw=black, very thick]	 (-0.4,0.5) -- (-0.6,0.5);
	\draw[-][draw=black, very thick]	 (-0.6,0) -- (-0.6,0.5);
	\node [left] at (-0.6,0.25) {Low};
	
	
	\node [below] at (5,0) {In: 74HC02};
	\draw[-][draw=gray, line width=2mm] (5,0) -- (5,1.35);
	\draw[-][draw=gray, line width=2mm] (5,3.15) -- (5,4.5);
	\draw[-][draw=black, very thick] (5.2,1.35) -- (4.8,1.35) node[label=left:1.35 V](){};
	\node [above] at (5.2,4.5) {4.5 V};
	
	\draw[-][draw=black, very thick] (5,0) -- (5,4.5);
	\draw[-][draw=black, very thick] (4.9,2.25) -- (5.1,2.25) node[label=right:2.25 V](){};
	\draw[-][draw=black, very thick] (4.8,0) -- (5.2,0);
	\draw[-][draw=black, very thick] (4.8,4.5) -- (5.2,4.5);
	\draw[-][draw=black, very thick] (5.4,0) -- (5.6,0);
	\draw[-][draw=black, very thick] (5.2,3.15) -- (4.8,3.15);
	\draw[-][draw=black, very thick] (5.4,1.35) -- (5.6,1.35);
	\draw[-][draw=black, very thick] (5.6,0) -- (5.6,1.35);
	\node [right] at (5.6,0.675) {Low};

	\draw[-][draw=black, very thick] (5.4,3.15) -- (5.6,3.15);
		\draw[-][draw=black, very thick] (5.4,4.5) -- (5.6,4.5);
	\draw[-][draw=black, very thick] (5.6,3.15) -- (5.6,4.5);
	\node [right] at (5.6,3.825) {High};
	
	\node [left] at (4.8,3.35) {3.15 V};	
	
	\draw[pattern=custom north west lines, hatchcolor=red, hatchspread=6pt] (0,2.7) rectangle (5,3.15);	
	
\end{tikzpicture}
\end{center}
\caption{Comparación de tecnologías con LS a la salida y HC a la entrada.}
\label{fig:lshc}
\end{figure}

A la hora de conectar una compuerta con otra, es deseable que los rangos de valores validos de salida sean menores que los de entrada, ya que de esta forma se garantiza que cualquier salida sea interpretada adecuadamente por la siguiente etapa. Por consiguiente, de la Tabla (\ref{tabla:nm}) se destaca el valor negativo de $NM_{High}$ al colocar las compuertas de tecnología HC a la salida de una LS, detalle que se vuelve a observar en la Figura (\ref{fig:lshc}). Al conectar los dispositivos como se mencionó anteriormente, se pone en evidencia que existe la posibilidad de que tensiones de salida que se consideran altas caigan en un margen en el cual la siguiente compuerta las considera como valores imprecisos, es decir, que no actúa frente a estos. Particularmente, tensiones de salida desde $2.7 \ V$ hasta $3.15 \ V$ sin incluir, que son considerados como activos altos para la tecnología LS, no lo son para la HC. Por lo tanto, no es conveniente realizar dicha conexión, ya que se podría generar perdida de datos.

\subsection{Desarrollo del circuito}

\subsection{Mediciones}

\end{document}