\documentclass[a4paper]{article}
\usepackage[utf8]{inputenc}
\usepackage[spanish, es-tabla]{babel}

\usepackage{amsmath}
\usepackage{amsfonts}
\usepackage{amssymb}

\usepackage{float}
\usepackage{graphicx}

\usepackage{caption}
\usepackage{subcaption}
\captionsetup{compatibility=false}

\usepackage{multirow}
\setlength{\doublerulesep}{\arrayrulewidth}

\newcommand{\quotes}[1]{``#1''}
\newcommand\underrel[2]{\mathrel{\mathop{#2}\limits_{#1}}}

\usepackage{array}
\newcolumntype{C}[1]{>{\centering\let\newline\\\arraybackslash\hspace{0pt}}m{#1}}

\usepackage[american]{circuitikz}
\usepackage{xcolor}
\usepackage{fancyhdr}

\newlength{\stockheight}
\usepackage{hyperref}

\hypersetup{
    colorlinks=true,
    linkcolor=blue,
    filecolor=magenta,      
    urlcolor=blue,
    citecolor=blue,    
}

\urlstyle{same}

\pagestyle{fancy}
\fancyhf{}
\lhead{22.13 Electrónica III}
\rhead{Mechoulam, Lambertucci, Martorell, Londero}
\rfoot{Página \thepage}

\begin{document}
\subsection{Introducción}

En el presente ejercicio se procedio a medir los tiempos de propagación, rise y fall de una compuerta NOR del IC 74HC02 primero al vacio y luego implementando el siguiente circuito y distintas modificaciones a este último:

\begin{figure}[h]
    \centering
    \includegraphics{ImagenesEjercicio4/pend.jpg}
    \caption{Circuito a implementar}
\end{figure}

\subsection{Mediciones}

Se realizaron las mediciones con sus distintas cargas y a distintas frecuencias obteniendose los siguientes resultados:

\begin{table}[]
\centering
\begin{tabular}{|c|c|c|c|c|}
\hline
Caso & $tpd_{L-H}(ns)$ & $tpd_{H-L}(ns)$ & trise$(ns)$ & tfall$(ns)$ \\ \hline
Sin carga & 11.10 & 8.75 & 21.0 & 19.0 \\ \hline
Con carga & 12.30 & 9.45 & 22 & 19.8 \\ \hline
Sin carga (100 kHz) & 8.35 & 9.85 & 19.6 & 19.1 \\ \hline
Con carga (100 kHz) & 12.15 & 9.25 & 20 & 19.4 \\ \hline
Con carga y capacitores (100 kHz) & 15.85 & 12.60 & 25 & 25.1 \\ \hline
\end{tabular}
\end{table}




\end{document}