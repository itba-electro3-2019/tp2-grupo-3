\documentclass[a4paper]{article}
\usepackage[utf8]{inputenc}
\usepackage[spanish, es-tabla]{babel}

\usepackage[a4paper, footnotesep = 1cm, width=18cm, left=2cm, top=2.5cm, height=25cm, textwidth=18cm, textheight=25cm]{geometry}
%\geometry{showframe}

\usepackage{tikz}
\usepackage{amsmath}
\usepackage{amsfonts}
\usepackage{amssymb}
\usepackage{float}
\usepackage{graphicx}
\usepackage{caption}
\usepackage{subcaption}
\usepackage{multicol}
\usepackage{multirow}
\setlength{\doublerulesep}{\arrayrulewidth}
\usepackage{xcolor}

\usepackage{hyperref}
\hypersetup{
    colorlinks=true,
    linkcolor=blue,
    filecolor=magenta,      
    urlcolor=blue,
    citecolor=blue,    
}

\newcommand{\quotes}[1]{``#1''}
\usepackage{array}
\newcolumntype{C}[1]{>{\centering\let\newline\\\arraybackslash\hspace{0pt}}m{#1}}
\usepackage[american]{circuitikz}
\usepackage{fancyhdr}
\usepackage{units} 

\pagestyle{fancy}
\fancyhf{}
\lhead{22.13 Electrónica III}
\rhead{Mechoulam, Lambertucci, Martorell, Londero}
\rfoot{\center \thepage}

\begin{document}
\subsection{Introducción}

En esta sección se procedió a realizar el análisis de dos compuertas lógicas de distintas tecnologías, las cuales que consisten en una compuerta AND de tecnología TTL y una compuerta OR CMOS, conectadas de la siguiente forma
\begin{figure}[H]
\begin{center}
\begin{circuitikz}
	\node [american and port](A){TTL};
	\draw (A.out) to[short, -o] ++(1,0) node[label=right:$Port$](){};
	\draw (A.in 1) to[short, -o] ++(-1,0) node[label=left:$V_{CC}$](){};
	\draw (A.in 2) to[short, -*] ++(-0.5,0);
	
	\draw (A) to[open] ++(0,-2) node[american or port](O){\footnotesize{MOS}};
	\draw (O.out) to[short, -o] ++(1,0) node[label=right:$Port$](){};
	\draw (O.in 2) to[short] ++(-1,0) node[ground](){};
	\draw (O.in 1) to[short, -*] ++(-0.5,0);
\end{circuitikz}
\caption{Circuitos en vacío.}
\label{fig:circ-vacio}
\end{center}
\end{figure}

\subsection{Análisis compuerta AND Open Gate}

Para realizar este análisis se utilizó una de las 4 compuertas que brinda el integrado \href{http://www.ti.com/lit/ds/symlink/sn74s08.pdf}{SN74S08}. Como es una compuerta AND, y una de sus entradas ya esta conectada a $V_{CC}$, la señal de salida dependerá solo del valor que tenga la señal en esa sola entrada. Ahora, dejando al vacío esa entrada, se puede observar una tensión continua de valor aproximada $1.45 \ V$, que corresponde al rango de valores que la compuerta considera como indeterminados, obteniéndose así a la salida un 1 lógico. Esto ocurre debido a que se esta dejando al vacío el emisor del transistor al que le corresponde esa entrada, por lo tanto dicho transistor se encuentra al corte, lo que hace que a la salida siempre se vea dicho valor.

\subsection{Análisis compuerta OR Open Gate}

De forma análoga al caso anterior, se utilizó una de las compuertas lógicas que brinda el integrado \href{http://www.ti.com/lit/ds/symlink/cd4071b.pdf}{CD4071}, pero en este caso, se conecto uno de sus pines de entrada a $GND$, dejando el otro abierto. Es así que el valor que se ve a la salida depende únicamente del valor de la entrada que se dejó abierta. Como esta compuerta es de tecnología MOS, conectándose a su entrada el GATE de un transistor de este mismo tipo, y debido a la gran impedancia de entrada que poseen, actúan como antena, lo que las hace mas susceptibles a cualquier señal de ruido que se encuentre presente. Teóricamente, si dicha señal de ruido llega a poseer una tensión lo suficientemente alta como para superar la $V_{TH}$ del transistor, este se activa y produce una oscilación a la salida de la compuerta. Las distintas pruebas realizadas para poder observar este fenómeno resultaron inconclusas ya que, ni dejando al vacío completamente la entrada, ni tocándola con la mano, se alcanzó una señal de ruido lo suficientemente grande para que ocasionará dicho fenómeno.

\begin{center}
	\huge{\textcolor{red}{\textbf{Alta impedancia...}}}
\end{center}

\subsection{Análisis ambas compuertas conectadas entre si}

Luego, se conecto los circuitos de la siguienta manera:
\begin{figure}[H]
\begin{center}
\begin{circuitikz}
	\node [american and port](A){};
	\draw (A.in 1) to[short, -o] ++(-1,0) node[label=left:$V_{CC}$](){};
	\draw (A.in 2) to[short, -o] ++(-1,0) node[label=left:$Input$](){};
	
	\draw (A.out) |- ++(0,-1) node[american or port, anchor=in 1](O){};
	\draw (O.out) to[short, -o] ++(1,0) node[label=right:$Port$](){};
	\draw (O.in 2) to[short] ++(-0.5,0) node[ground](){};
\end{circuitikz}
\caption{Conexión AND a la entrada de la OR.}
\label{fig:circ-andor}
\end{center}
\end{figure}

La salida de este, por lo analizado en los anteriormente, solo depende de la señal de entrada que se utiliza. Analizando las hojas de datos de ambos integrados y utilizando una alimentación $V_{DD}= 4.5 \ V$, se obtiene que la tensión mínima de la salida en estado alto es $V_{OH}=2.5 \ V$, la cual cae en el rango de valores indeterminados para la OR, siendo esta $V_{IL}= 3.15 \ V$ en el peor de los casos. Es así que se puede ocasionar que, a pesar de que la salida de la AND sea HIGH, en la salida total del circuito se vea un 0 lógico.

\subsection{Solución al problema}

Una solución al problema mencionado anteriormente se basa en utilizar un circuito llamado Level Shifter, el cual se puede fabricar utilizando un transistor PNP y un par de resistencias. Este circuito toma la salida de la primer compuerta y, en el caso de que está sea HIGH, lleva dicho valor a un nivel de tensión más alto para que así la compuerta siguiente pueda tomar correctamente el valor que debe recibir. Dicha solución se implemento al circuito anterior y se pudo solucionar el problema.
\begin{figure}[H]
\begin{center}
\begin{circuitikz}
	\node [pnp](pnp){};
	\draw (pnp.E) to[R, -o] ++(0,2) node[label=left:$V_{CC}$](){};
	\draw (pnp.C) to[short] ++(0,-0.5) node[ground](){};
	\draw (pnp.B) to[R, -o] ++(-2,0) node[label=left:$Input$](){};
\end{circuitikz}
\caption{Implementación del level shifter.}
\label{fig:levelshifter}
\end{center}
\end{figure}

\end{document}