\documentclass[a4paper]{article}
\usepackage[utf8]{inputenc}
\usepackage[spanish, es-tabla]{babel}

\usepackage{amsmath}
\usepackage{amsfonts}
\usepackage{amssymb}

\usepackage{float}
\usepackage{graphicx}

\usepackage{caption}
\usepackage{subcaption}
\captionsetup{compatibility=false}

\usepackage{multirow}
\setlength{\doublerulesep}{\arrayrulewidth}

\newcommand{\quotes}[1]{``#1''}
\newcommand\underrel[2]{\mathrel{\mathop{#2}\limits_{#1}}}

\usepackage{array}
\newcolumntype{C}[1]{>{\centering\let\newline\\\arraybackslash\hspace{0pt}}m{#1}}

\usepackage[american]{circuitikz}
\usepackage{xcolor}
\usepackage{fancyhdr}

\newlength{\stockheight}
\usepackage{hyperref}

\hypersetup{
    colorlinks=true,
    linkcolor=blue,
    filecolor=magenta,      
    urlcolor=blue,
    citecolor=blue,    
}

\urlstyle{same}

\pagestyle{fancy}
\fancyhf{}
\lhead{22.13 Electrónica III}
\rhead{Mechoulam, Lambertucci, Martorell, Londero}
\rfoot{Página \thepage}

\begin{document}
\subsection{Introducción}

En esta sección se procedió a realizar el análisis de dos compuertas lógicas de distintas tecnologías, las cuales que consisten en una compuerta AND de tecnología TTL y una compuerta OR CMOS, conectadas de la siguiente forma

\begin{center}
	\huge{\textcolor{red}{\textbf{De qué forma?}}}
\end{center}

\subsection{Análisis compuerta AND Open Gate}

Para realizar este análisis se utilizó una de las 4 compuertas que brinda el integrado \href{http://www.ti.com/lit/ds/symlink/sn74s08.pdf}{SN74S08}. Como es una compuerta AND, y una de sus entradas ya esta conectada a $V_{CC}$, la señal de salida dependerá solo del valor que tenga la señal en esa sola entrada. Ahora, dejando al vació esa entrada, se puede observar que el valor que se obtiene a la salida corresponde a un 1 lógico. Esto ocurre debido a que se esta dejando al vacío el emisor del transistor al que le corresponde esa entrada, por lo tanto dicho transistor se encuentra al corte, lo que hace que a la salida siempre se vea dicho valor.

\subsection{Análisis compuerta OR Open Gate}

De forma análoga al caso anterior, se utilizó una de las compuertas lógicas que brinda el integrado \href{http://www.ti.com/lit/ds/symlink/cd4071b.pdf}{CD4071}, pero en este cas,o se conecto uno de sus pines de entrada a $GND$, dejando el otro abierto. Es así que el valor que se ve a la salida depende únicamente del valor de la entrada que se dejó abierta. Como la alta impedancia de

\begin{center}
	\huge{\textcolor{red}{\textbf{Alta impedancia...}}}
\end{center}


\subsection{Análisis ambas compuertas conectadas entre si}

Luego, se conecto los circuitos de la siguienta manera:
\begin{figure}[h]
    \centering
    \includegraphics{ImagenesEjercicio5/pend.jpg}
    \caption{Conexión de la AND con la OR.}
\end{figure}

La salida de este, por lo analizado en los anteriormente, solo depende de la señal de entrada que se utiliza. Analizando las hojas de datos de ambos integrados y utilizando una alimentación $V_{DD}= 4.5 \ V$, se obtiene que la tensión mínima de la salida en estado alto es $V_{OH}=2.5 \ V$, la cual cae en el rango de valores indeterminados para la OR, siendo esta $V_{IL}= 3.15 \ V$ en el peor de los casos. Es así que se puede ocasionar que, a pesar de que la salida de la AND sea HIGH, en la salida total del circuito se vea un 0 lógico.

\subsection{Solución al problema}

Una solución al problema mencionado anteriormente se basa en utilizar un circuito llamado Level Shifter, el cual se puede fabricar utilizando un transistor PNP y un par de resistencias. Este circuito toma la salida de la primer compuerta y, en el caso de que está sea HIGH, lleva dicho valor a un nivel de tensión más alto para que así la compuerta siguiente pueda tomar correctamente el valor que debe recibir.

\begin{figure}[h]
    \centering
    \includegraphics{ImagenesEjercicio5/pend.jpg}
    \caption{Implementación del level shifter}
\end{figure}

\end{document}